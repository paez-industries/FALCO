%%%% Alcance
\chapter{Alcance}
%% FALTA TERM¡NAR 
Con este trabajo se busca desarrollar una herramienta sencilla, modular y escalable que le permita al usuario adquirir de una manera más rápida, efectiva y confiable los datos necesarios para su investigación, además de esto pueda seguir adquiriendo y analizando dichos datos sin estar presente en el laboratorio. Para cumplir con lo anterior se establece las funciones básicas del sistema con los siguientes puntos a seguir:   

\begin{description}

\item[Construcción del switch:] Teniendo como una de las primeras necesidades, la precisión en cuanto a intervalos de tiempo de entrada y salida de operación de las cargas para la toma de datos, se hace necesario la construcción de un switch que permita ser controlado y programado de forma tal que logre sincronizarse con la fuente de generación chroma 61511 y las tarjetas de adquicisión de datos NI 9225, NI 9227, NI 9239 y NI cDAQ-9172 a través de una plataforma.

\item[Desarrollo de la plataforma de control:] como se mencionó antes, la sincronización de los diferentes subsistemas que operan en la adquisición de los datos es de vital importancia. Para esto se desarrollará una plataforma (sotfware) que permita controlar de forma fácil y sencilla los elementos antes mencionados, dicha plataforma tendrá las facultades de poder operar remotamente cumpliendo con uno de los objetivos ya mencionados.      

\item[Diseño y construcción de un sistema de protección]: Ya que los elementos que se trabajan en el laboratorio operan con tensiones y corrientes considerables y algunos de ellos son sensibles a sobre-tensiones y picos de corriente, además, en vista de que el sistema en conjunto pueda operar sin la presencia del usuario, se necesita un sistema de protección eléctrico que pueda funcionar de manera autónoma y reaccionar adecuadamente frente a los posibles fallos que se presenten protegiendo los elementos de laboratorio antes mencionados.  

 
 
  
  
\end{description} 