\documentclass[a4paper,oneside,12pt]{book}
\usepackage[spanish,es-tabla]{babel}
\usepackage{helvet}
\usepackage{courier}
\usepackage[T1]{fontenc}
\usepackage[utf8]{inputenc} %Para las tildes
\usepackage{geometry}
%\geometry{verbose,letterpaper,tmargin=50pt,bmargin=2cm,lmargin=2cm,rmargin=2cm,headheight=20pt,headsep=1cm,footskip=30pt}
\geometry{verbose,letterpaper,tmargin=1in,bmargin=1in,lmargin=1in,rmargin=1in,headheight=25pt,headsep=1cm,footskip=30pt}%márgenes para las normas APA
\usepackage{setspace}
\usepackage{graphicx} %Para las gráficas
\usepackage{multirow,colortbl}
\usepackage[dvipsnames]{xcolor}
\usepackage[colorlinks=true,linkcolor=black,urlcolor=blue,citecolor=RoyalBlue,urlcolor=blue]{hyperref}


\pdfstringdefDisableCommands{%Para eliminar los errores de hyperref en los capitulos
  \let\enspace\empty  % this causes the warning for \kern
  \let\noindent\empty }% this causes the warning for \indent
  
%\usepackage[colorlinks,bookmarks=true]{hyperref}
\usepackage{amsfonts} %Para simbolos y símbolo numeros reales
\usepackage{cite} % para contraer referencias
\usepackage{amsmath}

%\pagestyle{uheadings}
\setcounter{tocdepth}{3}
\usepackage{longtable}
\usepackage{pifont}
\onehalfspacing
\usepackage{fancyhdr}
\pagestyle{fancy}

\setcounter{secnumdepth}{3} %para que ponga 1.1.1.1 en subsubsecciones
\setcounter{tocdepth}{3} % para que ponga subsubsecciones en el indice

%\usepackage{anysize}%Para cambiar los márgenes
%Para que se habilite la opción de modificar las márgenes en las páginas
\newenvironment{changemargin}[2]{% 
\begin{list}{}{% 
\setlength{\topsep}{0pt}% 
\setlength{\leftmargin}{#1}% 
\setlength{\rightmargin}{#2}% 
\setlength{\listparindent}{\parindent}% 
\setlength{\itemindent}{\parindent}% 
\setlength{\parsep}{\parskip}% 
}% 
\item[]}{\end{list}} 

%% se generan campos para facilitar la edición del archivo
\def\autoruno#1{\gdef\insertautoruno{#1}\gdef\@authoruno{#1}}
\def\autordos#1{\gdef\insertautordos{#1}\gdef\@authordos{#1}}
\def\correouno#1{\gdef\insertcorreouno{#1}}
\def\correodos#1{\gdef\insertcorreodos{#1}}
\def\codigouno#1{\gdef\insertcodigouno{#1}}
\def\codigodos#1{\gdef\insertcodigodos{#1}}
\def\titulo#1{\gdef\inserttitulo{#1}\gdef\@titulo{#1}}
\def\director#1{\gdef\insertdirector{#1}}
\def\cargodirector#1{\gdef\insertcargodirector{#1}}
\def\correodirector#1{\gdef\insertcorreodirector{#1}}
\def\codirector#1{\gdef\insertcodirector{#1}}
\def\cargocodirector#1{\gdef\insertcargocodirector{#1}}
\def\correocodirector#1{\gdef\insertcorreocodirector{#1}}

% % Cambiar el nombre índicie de cuadros por indicie de tablas
% \renewcommand{\listtablename}{Índice de tablas}
% \renewcommand{\tablename}{Tabla}